%* glpk07.tex *%

\chapter{Installing GLPK on Your Computer}
\label{install}

\section{Downloading the distribution tarball}

The distribution tarball of the most recent version of the GLPK
package can be found on \url{http://ftp.gnu.org/gnu/glpk/} [via http]
and \url{ftp://ftp.gnu.org/gnu/glpk/} [via FTP]. It can also be found
on one of the FTP mirrors; see \url{http://www.gnu.org/prep/ftp.html}.
Please use a mirror if possible.

To make sure that the GLPK distribution tarball you have downloaded is
intact you need to download the corresponding `\verb|.sig|' file and
run a command like this:

\begin{verbatim}
   gpg --verify glpk-4.38.tar.gz.sig
\end{verbatim}

\noindent
If that command fails because you do not have the required public key,
run the following command to import it:

\begin{verbatim}
   gpg --keyserver keys.gnupg.net --recv-keys 5981E818
\end{verbatim}

\noindent
and then re-run the previous command.

\section{Unpacking the distribution tarball}

The GLPK package (like all other GNU software) is distributed in the
form of packed archive. This is one file named \verb|glpk-X.Y.tar.gz|,
where {\it X} is the major version number and {\it Y} is the minor
version number.

In order to prepare the distribution for installation you need to copy
the GLPK distribution file to a working subdirectory and then unpack
and unarchive the distribution file with the following command:

\begin{verbatim}
   tar zx < glpk-X.Y.tar
\end{verbatim}

\newpage

\section{Configuring the package}

After unpacking and unarchiving the GLPK distribution you should
configure the package,\linebreak i.e. automatically tune it for your
platform.

Normally, you should just \verb|cd| to the subdirectory \verb|glpk-X.Y|
and run the configure script, e.g.

\begin{verbatim}
   ./configure
\end{verbatim}

The `\verb|configure|' shell script attempts to guess correct values
for various system-dependent variables used during compilation. It uses
those values to create a `\verb|Makefile|' in each directory of the
package. It also creates file `\verb|config.h|' containing
platform-dependent definitions. Finally, it creates a shell script
`\verb|config.status|' that you can run in the future to recreate the
current configuration, a file `\verb|config.cache|' that saves the
results of its tests to speed up reconfiguring, and a file
`\verb|config.log|' containing compiler output (useful mainly for
debugging `\verb|configure|').

Running `\verb|configure|' takes about a minute. While it is running,
it displays some informational messages that tell you what it
is doing. If you don't want to see these messages, run
`\verb|configure|' with its standard output redirected to
`\verb|dev/null|'; for example, `\verb|./configure > /dev/null|'.

By default both static and shared versions of the GLPK library will be
compiled. Compilation of the shared librariy can be turned off by
specifying the `\verb|--disable-shared|' option to `\verb|configure|':

\begin{verbatim}
   ./configure --disable-shared
\end{verbatim}

\noindent
If you encounter problems building the library try using the above
option, because some platforms do not support shared libraries.

The GLPK package has some optional features listed below. By default
all these features are disabled. To enable a feature the corresponding
option should be passed to the configure script.

\verb|--with-gmp         | Enable using the GNU MP bignum library

This feature allows the exact simplex solver to use the GNU MP bignum
library. If it is disabled, the exact simplex solver uses the GLPK
bignum module, which provides the same functionality as GNU MP, however,
it is much less efficient.

For details about the GNU MP bignum library see its web page at
\url{http://gmplib.org/}.

\verb|--enable-dl        | The same as `\verb|--enable-dl=ltdl|'

\verb|--enable-dl=ltdl   | Enable shared library support (GNU)

\verb|--enable-dl=dlfcn  | Enable shared library support (POSIX)

Currently this feature is only needed to provide dynamic linking to
ODBC and MySQL shared libraries (see below).

For details about the GNU shared library support see the manual at
\url{http://www.gnu.org/software/libtool/manual/}.

\verb|--enable-odbc      |
Enable using ODBC table driver (\verb|libiodbc|)

\verb|--enable-odbc=unix |
Enable using ODBC table driver (\verb|libodbc|)

This feature allows transmitting data between MathProg model objects
and relational databases accessed through ODBC.

For more details about this feature see the supplement ``Using Data
Tables in the GNU MathProg Modeling Language'' (\verb|doc/tables.pdf|).

\verb|--enable-mysql     |
Enable using MySQL table driver (\verb|libmysql|)

This feature allows transmitting data between MathProg model objects
and MySQL relational databases.

For more details about this feature see the supplement ``Using Data
Tables in the GNU MathProg Modeling Language'' (\verb|doc/tables.pdf|).

\section{Compiling the package}

Normally, you can compile (build) the package by typing the command:

\begin{verbatim}
   make
\end{verbatim}

\noindent
It reads `\verb|Makefile|' generated by `\verb|configure|' and performs
all necessary jobs.

If you want, you can override the `\verb|make|' variables \verb|CFLAGS|
and \verb|LDFLAGS| like this:

\begin{verbatim}
   make CFLAGS=-O2 LDFLAGS=-s
\end{verbatim}

To compile the package in a different directory from the one containing
the source code, you must use a version of `\verb|make|' that supports
`\verb|VPATH|' variable, such as GNU `\verb|make|'. `\verb|cd|' to the
directory where you want the object files and executables to go and run
the `\verb|configure|' script. `\verb|configure|' automatically checks
for the source code in the directory that `\verb|configure|' is in and
in `\verb|..|'. If for some reason `\verb|configure|' is not in the
source code directory that you are configuring, then it will report
that it can't find the source code. In that case, run `\verb|configure|'
with the option `\verb|--srcdir=DIR|', where \verb|DIR| is the
directory that contains the source code.

Some systems require unusual options for compilation or linking that
the `\verb|configure|' script does not know about. You can give
`\verb|configure|' initial values for variables by setting them in the
environment. Using a Bourne-compatible shell, you can do that on the
command line like this:

\begin{verbatim}
   CC=c89 CFLAGS=-O2 LIBS=-lposix ./configure
\end{verbatim}

\noindent
Or on systems that have the `\verb|env|' program, you can do it like
this:

\begin{verbatim}
   env CPPFLAGS=-I/usr/local/include LDFLAGS=-s ./configure
\end{verbatim}

Here are the `\verb|make|' variables that you might want to override
with environment variables when running `\verb|configure|'.

For these variables, any value given in the environment overrides the
value that `\verb|configure|' would choose:

\verb|CC      | C compiler program. The default is `\verb|cc|'.

\verb|INSTALL | Program used to install files. The default value is
`\verb|install|' if you have it,\\
\verb|           | otherwise `\verb|cp|'.

For these variables, any value given in the environment is added to the
value that `\verb|configure|' chooses:

\verb|DEFS    | Configuration options, in the form
`\verb|-Dfoo -Dbar| \dots'.

\verb|LIBS    | Libraries to link with, in the form
`\verb|-lfoo -lbar| \dots'.

\section{Checking the package}

To check the package, i.e. to run some tests included in the package,
you can use the following command:

\begin{verbatim}
   make check
\end{verbatim}

\section{Installing the package}

Normally, to install the GLPK package you should type the following
command:

\begin{verbatim}
   make install
\end{verbatim}

\noindent
By default, `\verb|make install|' will install the package's files in
`\verb|usr/local/bin|', `\verb|usr/local/lib|', etc. You can specify an
installation prefix other than `\verb|/usr/local|' by giving
`\verb|configure|' the option `\verb|--prefix=PATH|'. Alternately, you
can do so by consistently giving a value for the `\verb|prefix|'
variable when you run `\verb|make|', e.g.

\begin{verbatim}
   make prefix=/usr/gnu
   make prefix=/usr/gnu install
\end{verbatim}

After installing you can remove the program binaries and object files
from the source directory by typing `\verb|make clean|'. To remove all
files that `\verb|configure|' created (`\verb|Makefile|',
`\verb|config.status|', etc.), just type `\verb|make distclean|'.

The file `\verb|configure.ac|' is used to create `\verb|configure|' by
a program called `\verb|autoconf|'. You only need it if you want to
remake `\verb|configure|' using a newer version of `\verb|autoconf|'.

\section{Uninstalling the package}

To uninstall the GLPK package, i.e. to remove all the package's files
from the system places, you can use the following command:

\begin{verbatim}
   make uninstall
\end{verbatim}

%* eof *%
